%! aboutthismanual.tex $ this file belongs to the Molcas repository $*/

\section{The \molcas\ Manual}
\label{sec:about_this_manual}

%================================================================
\subsection{Manual in Four Parts}
\label{sec:who_should_read}

This manual is designed for use with the {\it ab initio} Quantum
chemistry software package \molcas\ \molcasversion\ developed at the
by the world-wide \molcas\ team where its base and origin is the
Department of Theoretical Chemistry, Lund University, Sweden.  \molcas\ is designed for use
by Theoretical Chemists and requires knowledge of the
Chemistry involved in the calculations in order to produce and
interpret the results correctly.  The package can be moderately difficult to use
because of this `knowledge requirement', but the results are often more
meaningful than those produced by \textbf{\textit{blackbox}} packages which may not be
sufficiently chemically precise in either input or output.

The \molcas\ manual is divided in four parts to facilitate its use.

\begin{enumerate}
\item The \textbf{\textit{\molcas\ Installation Guide}} describes simple and more complex aspects on how to install, tailor, and
control the \molcas\ package.

\item The \textbf{\textit{Short Guide to \molcas }} is a brief introductory guide which addresses the needs of the novice and intermediate users
and is designed for all those who want to start using \molcas\ as soon as possible. 
Only basic environment definitions, simple input examples, and minimal description of output results are included in the short guide.

Two types of introductory tutorials are given in the short guide: problem-based and program-specific.
\begin{enumerate}
\item Problem-based tutorials are exercises focused on solving a simple Quantum
Chemical project and contain all the required input files. Examples include 
computing electronic energy of a molecule at different levels of
theory, optimizing the geometry of a molecule, calculating the transition state in the ground
state of a chemical system, and computing an excited state.
The input files for this section can be found in the directory {\$MOLCAS/doc/samples/problem\_based\_tutorials}.
These examples are also employed in \molcas\ workshops that the \molcas\ team has organized in recent years.

\item Another type of tutorial is designed for the first-time user to provide an understanding of program modules 
contained in \molcas\ include simple, easy-to-follow examples for many of these modules.
\end{enumerate}

The systems covered in the short guide are not necessarily calculated with most suitable methods or produce highly significant results,
but provide both several tips for the beginner and actual input file formats.

The \textbf{\textit{Short Guide to \molcas}} can be independently printed as a booklet.
\item The \textbf{\textit{\molcas\ User's Guide}} contains a complete listing of the input
keywords for each of the program modules and a information regarding
files used in each calculation. Here the user will find all keywords that can be
used together with a specific program and thus how to set up the input for a
\molcas\ run.

\item \textit{\textbf{Advanced Examples}} and \textbf{\textit{Annexes}} include outlines of
actual research performed using \molcas. 

The approach to a research project is outlined including input files and shell scripts. More
importantly, however, the value of the calculations is evaluated and
advanced features of \molcasversion\ are used and explained to improve the
value of the results.
\end{enumerate}

The complete manual is available on the net in HTML and PDF formats \\
(\MolcasWWW).

%================================================================
\subsection{Notation}
\label{sec:notation}

For clarity, some words are printed using special typefaces.

\begin{itemize}
\item
Keywords, i.e. words used in input files, are typeset in
the small-caps typeface, for example \keyword{EndOfInput}.
\item
Programs (or modules) are typeset in the teletype typeface.
This will eliminate some potential confusion. For example,
when discussing the RASSCF method, regular uppercase letters
are used, while the program will look like \program{RASSCF}.
\item
Files are typeset in the slanted teletype typeface, like
\file{InpOrb}.
\item 
Commands, unix or other, are typeset in a sans serif typeface,
like \command{ln -fs}.
\item
Complete examples, like input files, shell scripts, etc,
are typeset in the teletype typeface.
\end{itemize}
